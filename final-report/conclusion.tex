\section{Conclusion}
Overall, we believe that our project was a great success, and we enjoyed the opportunity to develop a product from the inception of our initial idea through to a functional prototype. The team worked well together, and we take away from this project new technical and interpersonal skills. We found that dealing with unpredictable changes to customer requirements was challenging, but we mitigated any adverse consequences through frequent and early communication with the customer. 

In contrast to other teams, our simulation operated on a more macroscopic scale — not only because ships need to travel much longer distances, but also because we felt this approach would accurately convey the benefits the system can deliver and emphasise its potential global impact. Our simulation engine used state of the art libraries to generate a rich 3D visualisation, and we elected to logically separate the back-end and front-end to ensure that we were using the best technologies for each component. This logical separation has the added benefit that our visualisation tools are kept completely separate from our control software, making it easier to implement the control mechanisms in hardware at a later stage. Our analysis allowed us to conclude that, with little doubt, the OceanX system would be more efficient, able to handle a 2x increase in cargo volumes over the next 30 years, and move the industry toward greener, safer transportation.