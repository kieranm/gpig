\section{System Prototype}
Our prototype will demonstrate three key aspects of the system:
\begin{itemize}[noitemsep,topsep=0pt]
	\item the interaction of autonomous ships with coastal ports;
	\item collaboration and planning during transport;
	\item interaction of autonomous ships and offshore port.
\end{itemize}

Some of the key advantages of the proposed system are increased throughput and decreased congestion at coastal ports. To demonstrate this, we will produce a simulation showing how autonomous ships will dock with existing coastal ports to load and unload goods. We will demonstrate the control systems required to enable the shuttle ships to avoid other ships and obstacles, and to guide them into an unloading facility at the port. In addition, autonomous ships will collaborate to minimise the time they spend in the dock -- queueing in a logical order in the event that no unloading facilities are available. Our simulation will also include a comparison to traditional transport systems to further motivate the autonomous approach.

To demonstrate the fleet’s ability to coordinate at sea and plan routes, we will introduce adverse weather conditions and the presence of other physical obstacles. An early warning system will be implemented through ship to ship communication: when an obstacle is detected, the ship will use the communication system to inform nearby vessels. The route planning system for each ship will recalculate the optimal path, taking into account the obstacles detected by the fleet. We will compare the collaborative approach with a non communicating fleet, to exhibit the benefits of ship to ship communication.

In addition, we will examine the core functionality of the offshore port through its interaction with large intercontinental freight ships and smaller autonomous ships. In this simulation, cargo will be loaded and unloaded using the offshore port storage and retrieval system. The aim is to show the feasibility of the offshore port system, examining container throughput and efficiency of cargo routing.

Finally, we will perform two more simulations to demonstrate the impact of the system at national distribution and global distribution scales. The national level simulation will demonstrate the system’s potential to utilise existing small ports for more efficient transportation of goods on land. The global level simulation will show a network of these systems placed at key locations, we will compare the efficiency of the global system against the traditional shipping network.
