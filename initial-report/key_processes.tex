\section{Key Processes}
The team will follow agile practices by conducting regular `scrum’ meetings. The aim of these meetings is to gather system requirements, and break down the development of large, complex features into a series of actionable tasks. These tasks are each assigned a number of `story points’ depending on their complexity, added value, and thus the amount of time they will take to complete.

The use of kanban boards on GitHub to track the status of each task allows for the effective management of feature development, defect tracking and writing tasks. The tasks will be distributed among team members using a pair programming scheme, with the aim of ensuring the correctness and quality of initial code \cite{hannay2009effectiveness}. The persons responsible for each task will maintain and update the status of their tickets on the kanban board.

Progress will be presented to the team in the form of daily `stand ups’. The purpose of the stand ups is to ensure the team is focused, and consistently progressing towards its goals. Stand ups will take place online using Slack, where each team member will report the following:

\begin{itemize}[noitemsep,topsep=0pt]
	\item What progress they have made since the previous stand up;
	\item What tasks and goals they are currently working towards;
	\item What obstacles they anticipate facing, and what methods they will use to overcome them;
	\item What progress they aim to have achieved before the next stand up.
\end{itemize}

We will conduct unit tests in order to assess the functionality of each component. In addition to continual code review within each pair, code will be reviewed before being merged to the main branch. This is handled by the designated product owner, to ensure compatibility between functional components.

\section{Metrics}

In an agile development setting, measuring progress is often difficult. Hartmann and Dymond \cite{hartmann2006appropriate} identify 11 properties of effective agile measurements. It is important for metrics to be meaningful, and focus on highlighting delivered value as opposed to output quantity. This ensures that the metrics accurately represent the team's progress, and the quality of the product. In view of these properties, we opt to measure the following key performance indicators:

\begin{description}[topsep=0pt]
	\item [Story Points Completed per Day] -- represents the velocity of progress. More complex tasks are assigned more story points. Therefore, the number of story points completed per day acts as an accurate measure of how much work is actually being completed -- more so than quantitative output metrics such as lines of code \cite{davis2015agile}. Measuring on a per day basis allows for trends to quickly be established, despite the short project duration.
	\item [Estimated vs. Actual Time] -- represents the deviation from our initial effort estimates, plotted on a graph on a daily basis. Deviation from the estimated time has been shown to be an effective performance measure, and will act as an indicator that team efforts need distributing more effectively \cite{greening2015agile}.
	\item [Number and Severity of Faults per Day] -- represents the maturity and correctness of the system. By logging number of defects alone, we cannot obtain a representative measure of maturity. This is because a fault could be a simple one line code change, or it might require significant modifications to functionality. We therefore opt to log the severity of faults on a scale of Minor, Major and Critical -- with the aim of observing a decrease in severity of faults over time.
	\item [Team Satisfaction] -- represents the extent to which team members are satisfied with both their own, and team progress. Measured on a linear scale that ranges from 1 (very dissatisfied) to 5 (very satisfied). Team morale has been shown to provide an important indication as to how successful the final outcome will be \cite{prowarenessagile}.
\end{description}

The collection of these metrics will be aided by the use of kanban boards and a Gantt chart, allowing us to keep track of development time, story points, number of defects and fault severity. Soft metrics such as team satisfaction will be collected via an automated process in Slack.
