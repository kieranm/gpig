\section{Key Processes}
The team will follow agile practices by conducting regular `scrum’ meetings. The aim of these meetings is to gather system requirements, and break down the development of large, complex features into a series of actionable tasks. These tasks are each assigned a number of `story points’ depending on their complexity, added value, and thus the amount of time they will take to complete.

The use of kanban boards on GitHub to track the status of each task in the development pipeline allows for the effective management of feature development, issue tracking and writing tasks. The tasks will be distributed among team members, using a pair programming scheme with the aim of increasing the correctness and quality of initial code \cite{hannay2009effectiveness}. The persons responsible for each task will maintain and update the status of their tickets on the kanban board.

Progress will be presented to the team in the form of daily `stand ups’, which will ensure the team is focused, and consistently progressing towards its goals. Stand ups will take place online using Slack, where each team member will report the following:

\begin{itemize}[noitemsep,topsep=0pt]
	\item What progress they have made since the previous stand up;
	\item What tasks and goals they are currently working towards;
	\item What obstacles they anticipate facing, and what methods they will use to overcome them;
	\item What progress they aim to have achieved before the next stand up.
\end{itemize}

We will conduct unit tests in order to assess the functionality of each component to ensure correct integration with other product interfaces. Alongside continual code review within each pair, code will be reviewed before merged to the main branch. This is handled by the designated product owner, to ensure compatibility between functional components.

\section{Metrics}
