\section{Problem Analysis}
\label{problem_analysis}

With more than 90\% of all international trade conducted by sea, the shipping industry plays a crucial role in the modern interconnected world \cite{IMOfacts}. Over the past 50 years, the size of the world shipping fleet has increased from 36,000 to over 100,000 \cite{IMOfacts}. In order to maximise efficiency of transportation \cite{sizematters}, the capacities of these ships have increased substantially. In 1956, for instance, a typical container ship had a capacity of 101 Twenty-foot Equivalent Units (TEUs) \cite{ftarticle}. By comparison, ships operating in 2015 have capacities of up to 20,000 TEUs \cite{sizematters}.

The rapid expansion in both fleet size and freight capacity raises many questions about the sustainability of the shipping industry. In 2016, freight rates (the wholesale price of shipping) fell to the lowest levels since the 2008 global recession. This caused Hanjin, a major operator in the shipping industry, to enter administration \cite{globalCrisis}. We argue that, if existing companies wish to meet future demand and survive further economic turmoil \cite{5trends}, the industry must adapt. 

Since the growth of seaborne trade shows little sign of slowing, we can expect to see further increases in both the number of ships and their capacities. This necessitates the development of coastal infrastructure to accommodate larger vessels, putting extra pressure on already taxed resources \cite{ftarticle}. At the same time, there will be increasing pressure from political bodies, such as the EU, to reduce the impact the industry is having on the environment \cite{eureport}. This is forcing the industry to explore `greener’ approaches, perhaps at the cost of more economical alternatives.

Collaboration and autonomy in shipping systems have the potential to mitigate these problems, and many techniques are in development today. For instance, Rolls-Royce plan to bring autonomous container ships into service by 2020 \cite{autoboats}. These ships will combine data from satellites, weather reports, and an array of LIDAR and infrared sensors to enable completely unmanned navigation. It is therefore reasonable to assume that by 2030, these techniques will be commonplace and can be incorporated into our solution.  

Autonomous ships generate a wealth of data, which can be leveraged by other ships and ports to make more informed decisions \cite{5trends}. The automotive industry is already exploring ways in which the data generated by autonomous vehicles can be applied to solve pressing problems. One proposed system collects broadcasted information from other vehicles, such as speed and position, to create a model of the environment and provide the driver with an early warning system for potential hazards \cite{c2c}. With development of these systems already in progress, we believe that the shipping industry must consider investment into collaborative and autonomous technologies.
