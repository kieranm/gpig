\section{Advantages and Benefits}

To be economically viable, maritime transportation relies heavily on economies of scale \cite{sizematters}. The number of days spent in port is a fixed cost which does not increase as a function of the number of transported units \cite{gkonis2010some}. To absorb these costs, ships have to increase in size. The current rate of growth, however, is not sustainable. Wright \cite{ftarticle} argues that ``the next step-up in size would impose such significant costs on ports that they would outweigh the advantages of moving cargo in ever-larger vessels”. The size of modern freight ships also significantly limits the number of ports in which they are able to dock \cite{sizematters, berth-efficiency}. By moving ports offshore, we enable the continued growth of freight ships, thus allowing shipping companies to sustain profits. 

For a given shipping journey, the expenses incurred by the shipping company are largely fixed irrespective of the number of containers transported. For this reason, it is more profitable to make the journey with the ship filled to its capacity. Our small autonomous shuttle ships allow for cargo from multiple low-demand ports to be coalesced at a single offshore port, reducing the likelihood that ships will be forced to make journeys only partially loaded. This approach is more economical, and also offers significant environmental benefits since fewer overall journeys need to be made to transport the same volume of cargo. 

By shifting traffic away from the shore, we reduce the need for coastal port expansion. Due to pre-existing surrounding structures such as buildings, the expansion of these ports cannot continue indefinitely. Since our autonomous shuttle ships are significantly smaller, they have more freedom in the destinations they serve. This means that deliveries can be distributed more evenly across smaller coastal ports, removing bottlenecks and reducing congestion. Furthermore, our autonomous shuttle ships can reach deeper inland, travelling down rivers and canals to take cargo shipments closer to their intended destination. This ultimately reduces the need for long-haul land transportation and lowers congestion and pollution on roads.

In addition, the ratio of surface area to mass of our drone ships, when compared to that of intercontinental cargo ships, makes solar-assisted propulsion feasible. The use of batteries enables the autonomous ships to operate at night and during bad weather. The shuttle ships will produce almost no harmful emissions, minimising their impact on the environment. Keeping polluting ships out to sea offers obvious public health benefits. Since these ships are not reliant on diesel fuel, their operation is resilient to fluctuations in oil price.

Autonomous ships are inherently invulnerable to attacks from pirates and other malicious parties, since there is no crew to hold to ransom and no possibility of cargo theft, assuming that the containers themselves are adequately secured. Theft of the vessel itself is more likely, but risky for any potential attackers due to the fact it will be tracked through GPS, and extensive anti-tamper measures will be put in place.

Finally, our fleet’s swarm intelligence mechanisms make it much easier for ships to avoid each other and resolve any congestion efficiently. This is very important given the large number of drone ships we anticipate will be necessary to service our offshore ports. Swarm behaviour also presents opportunities for faulty ships to receive assistance from other nearby ships -- allowing them to be safely returned to shore without the intervention of a human maintenance team.
